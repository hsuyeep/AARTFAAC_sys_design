\documentclass{article}
\usepackage{natbib}
\begin {document}
\title  {Referee  comments}
\maketitle

\textbf  {Editor  1:}

\textit{Given  the  similarity of  some  aspects  of  the system  to  previously
  published FPGA/GPU  correlator work,  please also add  references to  e.g. the
  CHIME, LEDA, \& PAPER experiment correlators.}\\

A new paragraph has been added to Sec. 1: ``Telescopes being
built for science  cases other than transient detection,  notably new generation
high redshift  $\lambda21cm$ experiments,  can also  employ a  hierarchical data
routing and  processing scheme similar to  ours.  The large number  of receiving
elements necessary for high sensitivity  wide field observations have resulted in
hybrid FPGA/GPU approaches to correlation  similar to ours.  Notable among these
are  the  LEDA  \citep{kocz2015digital}, CHIME  \citep{bandura2014canadian}  and
PAPER \citep{parsons2010precision} experiments.''\\

\textbf {Reviewer 1:}
\begin {enumerate}
  
\item  \textit{The  instrument  could  be better  motivated  by  a  quantitative
  comparison  of  the   AARTFAAC  specs  with  (a)   expected  transient  source
  populations (the authors mention the Swinbank et al. NCP transient, and FRBs),
  and (b) areas  of the cadence-sensitivity-resolution phase  space that haven't
  yet been probed. For example, the  authors could provide estimates of how many
  transients  like the  Swinbank  et al.  event could  be  detected, along  with
  estimates for  FRBs, emission from extrasolar  planets, the Jaeger et  al. VLA
  transient, etc.}

  The following  statement has  been added to  the introduction
  section:

  ''Among the  many sources  known and  speculated to be  emitting at  low radio
  frequencies, AARTFAAC would  be particularly sensitive to  brighter (and hence
  rarer) transients due  to its wide field of view  and continuous sky coverage.
  Solar type  II and III  bursts, usually the  brightest transients in  the sky,
  would be spatially resolved, allowing  their temporal evolution to be studied.
  Jovian Decametric (DAM) bursts are expected to be another source of transients
  bright enough  to be  regularly seen  by the AARTFAAC.   Detection of
  Jovian     synchrotron     emission  would be    possible     in     deeper     images
  \citep{girard2012jupiter,zarka2004fast}, allowing   the  tracking   of  their
  temporal and spectral variability.

  Transients of the  kind detected by \citep{stewart2016lofar},  with a duration
  of a  few minutes and a  flux density at 60MHz  of 15-25 Jy would  be the most
  intriguing  discovery from  the AARTFAAC.  Based on  their published  rates of
  $\sim1.5x10^{-5}  deg^{-2}$, and  assuming a  cosmological population  of such
  transients,  our  detection  rate  with  a  $3\sigma$  sensitivity  of  42  Jy
  (\ref{tab:afaac_specs}) is expected to be about $7 sky^{-1}day^{-1}$.
  
\item \textit{More detail  on how AARTFAAC could benefit  LOFAR operations would
  be appreciated (Page 2, paragraph 3).}

  The introduction section has a  new paragraph: ``The AARTFAAC, being commensal
  and operating in real-time, can also  be beneficial for LOFAR operations.  The
  system  can  provide  instantaneous  feedback   on  a  number  of  operational
  parameters  like the  health of  the system  at the  dipole level,  missing or
  corrupt timeslices,  directional information about RFI  sources, monitoring of
  RFI prone subbands, etc.  These can  be utilized to monitor and diagnose LOFAR
  issues.  Observational parameters like the presence  of a flaring Sun, or high
  ionospheric activity  can also be  instantaneously monitored.  if  provided as
  metadata  with  LOFAR  observations,  these  could  ease  the  segregation  of
  corrupted data stretches.''
  
  
\item  \textit {Will  the  AARTFAAC team  be  investigating polarised  transient
  emission,  which  may  be  important  in  overcoming  calibration  and  source
  confusion limits?}

  The current system can generate images from  all 4 cross and parallel hands in
  real-time. However, since our current focus is more on transient detection, we
  do not carry  out polarization calibration of the instrument  and analyse only
  the total power images. We plan to  bring in calibrated polarized imaging in a
  later version of the instrumentation.   Stokes-V images can be generated right
  now,  and are  expected to  beat  source confusion  limits due  to being  very
  sparsely populated, however, the current  system is limited to processing only
  Stokes-I images.
  

\item   \textit   {Are  the   performance   measures   in   Table  1   for   the
  highest-sensitivity part  of the primary  beam? Presumably the  sensitivity is
  confusion limited, and hence varies according to sky position?}\\
  
  Yes, the measure is  from the zenith, which is the most  sensitive part of the
  primary  beam.   The  array  is  confusion limited  after  about  30  secs  of
  integration \citep{prasad2014real},  not at  the 1sec cadences  reported here.
  The sky noise contributes more to the sensitivity variation than the confusion
  noise. Sensitivity variation is now mentioned  in the footnote of table 1.  We
  account for this noise variation in our analysis.
  

\item \textit {Some discussion of the motivation for searching for transients in
  image space, rather than visibility space, would be good. Why is this the best
  approach for AARTFAAC?}\\

  This was  indirectly alluded to in  the introduction section, which  has now a
  new paragraph  with the following  explanation: ``They typically have  a large
  number  of receiving  elements for  high sensitivities  and low  instantaneous
  sidelobes, resulting in high fidelity imaging from even snapshot observations.
  Image  domain  transient searches  allow  spatial  discrimination of  spurious
  sources of emission from sidelobes, and are inherently more robust against RFI
  than  beam-forming, especially  for wide-field  observations.  Further,  short
  cadence  transients like  FRBs are  expected  to be  scatter-broadened at  low
  frequencies to timescales  comparable to snapshot imaging  cadences, making it
  feasible to  detect their emission  in the  image domain.  Thus,  image domain
  searches are very attractive for wide fieldtransient monitors.''

\item  \textit {Does  the  6.25 MHz  band  have  to be  contiguous,  or can  the
  sub-bands be spaced out over the LBA band?}\\

  No, the 32  subbands can be independently chosen from  among the 512 available
  subbands. This  selection is independent from  the selection used by  LOFAR. A
  new  sentence has  been  added to  Sec.  3.2, paragraph  1:  ``This subset  of
  subbands can  be distributed among the  available 512 subbands in  any manner,
  from coverage  of contiguous subbands to  a sparse coverage across  the entire
  analog bandwidth.''

\item    \textit  {Page    1,   par.    2:   ``non-repeating''    $\rightarrow$
  ``repeating''}\\Done.
\end{enumerate}

\bibliographystyle{unsrt}
\bibliography{ref}

\end{document}
